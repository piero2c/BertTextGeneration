\documentclass{article}

\usepackage[utf8]{inputenc}

\usepackage[margin=1in]{geometry}
\usepackage[titletoc,title]{appendix}
\usepackage{booktabs}
\usepackage{amsmath,amsfonts,amssymb,mathtools}

\usepackage{graphicx,float}
\usepackage[ruled,vlined]{algorithm2e}
\usepackage{algorithmic}
\usepackage{tikz}
\usepackage[portuguese]{babel}


%\usemintedstyle{borland}

\usepackage[backend=biber]{biblatex}
\addbibresource{refs.bib}

% Title content
\title{Relatório - EP3}
\author{Piero Conti Kauffmann (8940810)}
\date{}

\begin{document}

\maketitle

% Introduction and Overview
\section{Modelos utilizados}

\subsection{Encoder-Decoder BiLSTM}

Pelas especificações da tarefa, o primeiro modelo proposto para o problema de geração automática de títulos  deve ser uma rede neural encoder-decoder com uma LSTM bidirecional ($\overrightarrow{g_{e}}$ e $\overleftarrow{g_{e}}$) como encoder acoplada a um mecanismo de atenção (Figura \ref{lstm_fig}). 

Na componente do decoder do modelo, escolhi incluir duas LSTMs unidirecionais em sequência. A primeira LSTM recebe o embedding do último token escrito no título, e é inicializada com os estados finais da rede bidirecional concatenados, portanto possuí o dobro de \textit{hidden units} de $\overrightarrow{g_{e}}$ e $\overleftarrow{g_{e}}$. Além disso, é na primeira LSTM do decoder em que é feito o cálculo dos vetores de contexto por meio do mecanismo de atenção do modelo, descrito adiante. Esses vetores são concatenados aos \textit{hidden states} da LSTM e são passados para a LSTM final como input, que finaliza a decodificação do próximo token da sequência.

\vspace{2em}

\begin{figure}[h]



\tikzset{every picture/.style={line width=0.65pt}} %set default line width to 0.75pt        

\begin{tikzpicture}[x=0.75pt,y=0.75pt,yscale=-1,xscale=1,scale=0.68]
%uncomment if require: \path (0,620); %set diagram left start at 0, and has height of 620

%Shape: Rectangle [id:dp06820819666280076] 
\draw  [fill={rgb, 255:red, 204; green, 239; blue, 255 }  ,fill opacity=1 ] (122,477) -- (192,477) -- (192,517) -- (122,517) -- cycle ;

%Shape: Rectangle [id:dp0898694390160586] 
\draw  [fill={rgb, 255:red, 204; green, 239; blue, 255 }  ,fill opacity=1 ] (218,477) -- (288,477) -- (288,517) -- (218,517) -- cycle ;

%Shape: Rectangle [id:dp8948734494987487] 
\draw  [fill={rgb, 255:red, 204; green, 239; blue, 255 }  ,fill opacity=1 ] (314,477) -- (384,477) -- (384,517) -- (314,517) -- cycle ;

%Shape: Rectangle [id:dp23757777085058995] 
\draw  [fill={rgb, 255:red, 204; green, 239; blue, 255 }  ,fill opacity=1 ] (410,477) -- (480,477) -- (480,517) -- (410,517) -- cycle ;

%Shape: Rectangle [id:dp2596001234599268] 
\draw  [fill={rgb, 255:red, 204; green, 239; blue, 255 }  ,fill opacity=1 ] (506,477) -- (576,477) -- (576,517) -- (506,517) -- cycle ;

%Shape: Rectangle [id:dp16562401384107073] 
\draw  [fill={rgb, 255:red, 204; green, 239; blue, 255 }  ,fill opacity=1 ] (28,477) -- (98,477) -- (98,517) -- (28,517) -- cycle ;

%Shape: Rectangle [id:dp3857197433233883] 
\draw   (720,477) -- (790,477) -- (790,517) -- (720,517) -- cycle ;

%Shape: Rectangle [id:dp09686078046699209] 
\draw   (816,477) -- (886,477) -- (886,517) -- (816,517) -- cycle ;

%Shape: Rectangle [id:dp17851992404896722] 
\draw   (626,477) -- (696,477) -- (696,517) -- (626,517) -- cycle ;

%Straight Lines [id:da8848937171112676] 
\draw    (62,476) -- (62,430) ;
\draw [shift={(62,428)}, rotate = 450] [color={rgb, 255:red, 0; green, 0; blue, 0 }  ][line width=0.75]    (10.93,-3.29) .. controls (6.95,-1.4) and (3.31,-0.3) .. (0,0) .. controls (3.31,0.3) and (6.95,1.4) .. (10.93,3.29)   ;
%Straight Lines [id:da47546724388980155] 
\draw    (156,477) -- (156,431) ;
\draw [shift={(156,429)}, rotate = 450] [color={rgb, 255:red, 0; green, 0; blue, 0 }  ][line width=0.75]    (10.93,-3.29) .. controls (6.95,-1.4) and (3.31,-0.3) .. (0,0) .. controls (3.31,0.3) and (6.95,1.4) .. (10.93,3.29)   ;
%Straight Lines [id:da22039276126083407] 
\draw    (253,477) -- (253,431) ;
\draw [shift={(253,429)}, rotate = 450] [color={rgb, 255:red, 0; green, 0; blue, 0 }  ][line width=0.75]    (10.93,-3.29) .. controls (6.95,-1.4) and (3.31,-0.3) .. (0,0) .. controls (3.31,0.3) and (6.95,1.4) .. (10.93,3.29)   ;
%Straight Lines [id:da8613756284673748] 
\draw    (349.33,477) -- (349.33,431) ;
\draw [shift={(349.33,429)}, rotate = 450] [color={rgb, 255:red, 0; green, 0; blue, 0 }  ][line width=0.75]    (10.93,-3.29) .. controls (6.95,-1.4) and (3.31,-0.3) .. (0,0) .. controls (3.31,0.3) and (6.95,1.4) .. (10.93,3.29)   ;
%Straight Lines [id:da49733654557561535] 
\draw    (444.67,477) -- (444.67,431) ;
\draw [shift={(444.67,429)}, rotate = 450] [color={rgb, 255:red, 0; green, 0; blue, 0 }  ][line width=0.75]    (10.93,-3.29) .. controls (6.95,-1.4) and (3.31,-0.3) .. (0,0) .. controls (3.31,0.3) and (6.95,1.4) .. (10.93,3.29)   ;
%Straight Lines [id:da8108597352855125] 
\draw    (541.33,477) -- (541.33,431) ;
\draw [shift={(541.33,429)}, rotate = 450] [color={rgb, 255:red, 0; green, 0; blue, 0 }  ][line width=0.75]    (10.93,-3.29) .. controls (6.95,-1.4) and (3.31,-0.3) .. (0,0) .. controls (3.31,0.3) and (6.95,1.4) .. (10.93,3.29)   ;
%Straight Lines [id:da12423693399685876] 
\draw    (661,477) -- (661,431) ;
\draw [shift={(661,429)}, rotate = 450] [color={rgb, 255:red, 0; green, 0; blue, 0 }  ][line width=0.75]    (10.93,-3.29) .. controls (6.95,-1.4) and (3.31,-0.3) .. (0,0) .. controls (3.31,0.3) and (6.95,1.4) .. (10.93,3.29)   ;
%Straight Lines [id:da9793440480468043] 
\draw    (756,477) -- (756,431) ;
\draw [shift={(756,429)}, rotate = 450] [color={rgb, 255:red, 0; green, 0; blue, 0 }  ][line width=0.75]    (10.93,-3.29) .. controls (6.95,-1.4) and (3.31,-0.3) .. (0,0) .. controls (3.31,0.3) and (6.95,1.4) .. (10.93,3.29)   ;
%Straight Lines [id:da7321987395268801] 
\draw    (849,476) -- (849,430) ;
\draw [shift={(849,428)}, rotate = 450] [color={rgb, 255:red, 0; green, 0; blue, 0 }  ][line width=0.75]    (10.93,-3.29) .. controls (6.95,-1.4) and (3.31,-0.3) .. (0,0) .. controls (3.31,0.3) and (6.95,1.4) .. (10.93,3.29)   ;
%Shape: Rectangle [id:dp8771114499487997] 
\draw  [fill={rgb, 255:red, 230; green, 230; blue, 230 }  ,fill opacity=1 ] (122,321.96) -- (192,321.96) -- (192,349.15) -- (122,349.15) -- cycle ;
%Shape: Rectangle [id:dp4427175459043031] 
\draw  [fill={rgb, 255:red, 230; green, 230; blue, 230 }  ,fill opacity=1 ] (218,321.96) -- (288,321.96) -- (288,349.15) -- (218,349.15) -- cycle ;
%Shape: Rectangle [id:dp4849337984465414] 
\draw  [fill={rgb, 255:red, 230; green, 230; blue, 230 }  ,fill opacity=1 ] (314,321.96) -- (384,321.96) -- (384,349.15) -- (314,349.15) -- cycle ;
%Shape: Rectangle [id:dp30544476974636803] 
\draw  [fill={rgb, 255:red, 230; green, 230; blue, 230 }  ,fill opacity=1 ] (410,321.96) -- (480,321.96) -- (480,349.15) -- (410,349.15) -- cycle ;
%Shape: Rectangle [id:dp9225647841495004] 
\draw  [fill={rgb, 255:red, 230; green, 230; blue, 230 }  ,fill opacity=1 ] (506,321.96) -- (576,321.96) -- (576,349.15) -- (506,349.15) -- cycle ;
%Shape: Rectangle [id:dp5225425679829652] 
\draw  [fill={rgb, 255:red, 230; green, 230; blue, 230 }  ,fill opacity=1 ] (27,321.96) -- (97,321.96) -- (97,349.15) -- (27,349.15) -- cycle ;
%Shape: Rectangle [id:dp9076370716655886] 
\draw  [fill={rgb, 255:red, 232; green, 232; blue, 232 }  ,fill opacity=1 ] (26,406) -- (936,406) -- (936,421) -- (26,421) -- cycle ;
%Straight Lines [id:da6972883310728979] 
\draw    (60.67,404) -- (60.67,358) ;
\draw [shift={(60.67,356)}, rotate = 450] [color={rgb, 255:red, 0; green, 0; blue, 0 }  ][line width=0.75]    (10.93,-3.29) .. controls (6.95,-1.4) and (3.31,-0.3) .. (0,0) .. controls (3.31,0.3) and (6.95,1.4) .. (10.93,3.29)   ;
%Straight Lines [id:da8667323617349345] 
\draw    (154.67,405) -- (154.67,359) ;
\draw [shift={(154.67,357)}, rotate = 450] [color={rgb, 255:red, 0; green, 0; blue, 0 }  ][line width=0.75]    (10.93,-3.29) .. controls (6.95,-1.4) and (3.31,-0.3) .. (0,0) .. controls (3.31,0.3) and (6.95,1.4) .. (10.93,3.29)   ;
%Straight Lines [id:da37193312511298227] 
\draw    (251.67,405) -- (251.67,359) ;
\draw [shift={(251.67,357)}, rotate = 450] [color={rgb, 255:red, 0; green, 0; blue, 0 }  ][line width=0.75]    (10.93,-3.29) .. controls (6.95,-1.4) and (3.31,-0.3) .. (0,0) .. controls (3.31,0.3) and (6.95,1.4) .. (10.93,3.29)   ;
%Straight Lines [id:da7354473166981947] 
\draw    (348,405) -- (348,359) ;
\draw [shift={(348,357)}, rotate = 450] [color={rgb, 255:red, 0; green, 0; blue, 0 }  ][line width=0.75]    (10.93,-3.29) .. controls (6.95,-1.4) and (3.31,-0.3) .. (0,0) .. controls (3.31,0.3) and (6.95,1.4) .. (10.93,3.29)   ;
%Straight Lines [id:da22937129148549373] 
\draw    (443.33,405) -- (443.33,359) ;
\draw [shift={(443.33,357)}, rotate = 450] [color={rgb, 255:red, 0; green, 0; blue, 0 }  ][line width=0.75]    (10.93,-3.29) .. controls (6.95,-1.4) and (3.31,-0.3) .. (0,0) .. controls (3.31,0.3) and (6.95,1.4) .. (10.93,3.29)   ;
%Straight Lines [id:da6672699657140333] 
\draw    (540,405) -- (540,359) ;
\draw [shift={(540,357)}, rotate = 450] [color={rgb, 255:red, 0; green, 0; blue, 0 }  ][line width=0.75]    (10.93,-3.29) .. controls (6.95,-1.4) and (3.31,-0.3) .. (0,0) .. controls (3.31,0.3) and (6.95,1.4) .. (10.93,3.29)   ;
%Straight Lines [id:da4106071108683027] 
\draw    (97,336.01) -- (119,336.01) ;
\draw [shift={(121,336.01)}, rotate = 180] [color={rgb, 255:red, 0; green, 0; blue, 0 }  ][line width=0.75]    (10.93,-3.29) .. controls (6.95,-1.4) and (3.31,-0.3) .. (0,0) .. controls (3.31,0.3) and (6.95,1.4) .. (10.93,3.29)   ;
%Straight Lines [id:da5470679817090385] 
\draw    (194,336.01) -- (216,336.01) ;
\draw [shift={(218,336.01)}, rotate = 180] [color={rgb, 255:red, 0; green, 0; blue, 0 }  ][line width=0.75]    (10.93,-3.29) .. controls (6.95,-1.4) and (3.31,-0.3) .. (0,0) .. controls (3.31,0.3) and (6.95,1.4) .. (10.93,3.29)   ;
%Straight Lines [id:da8737809883688727] 
\draw    (289,336.01) -- (311,336.01) ;
\draw [shift={(313,336.01)}, rotate = 180] [color={rgb, 255:red, 0; green, 0; blue, 0 }  ][line width=0.75]    (10.93,-3.29) .. controls (6.95,-1.4) and (3.31,-0.3) .. (0,0) .. controls (3.31,0.3) and (6.95,1.4) .. (10.93,3.29)   ;
%Straight Lines [id:da11848536863112558] 
\draw    (386,336.01) -- (408,336.01) ;
\draw [shift={(410,336.01)}, rotate = 180] [color={rgb, 255:red, 0; green, 0; blue, 0 }  ][line width=0.75]    (10.93,-3.29) .. controls (6.95,-1.4) and (3.31,-0.3) .. (0,0) .. controls (3.31,0.3) and (6.95,1.4) .. (10.93,3.29)   ;
%Straight Lines [id:da3090815034436545] 
\draw    (481,336.01) -- (503,336.01) ;
\draw [shift={(505,336.01)}, rotate = 180] [color={rgb, 255:red, 0; green, 0; blue, 0 }  ][line width=0.75]    (10.93,-3.29) .. controls (6.95,-1.4) and (3.31,-0.3) .. (0,0) .. controls (3.31,0.3) and (6.95,1.4) .. (10.93,3.29)   ;
%Shape: Rectangle [id:dp7192590390920259] 
\draw  [fill={rgb, 255:red, 230; green, 230; blue, 230 }  ,fill opacity=1 ] (122,289.56) -- (192,289.56) -- (192,316.75) -- (122,316.75) -- cycle ;
%Shape: Rectangle [id:dp08620349616309797] 
\draw  [fill={rgb, 255:red, 230; green, 230; blue, 230 }  ,fill opacity=1 ] (218,289.56) -- (288,289.56) -- (288,316.75) -- (218,316.75) -- cycle ;
%Shape: Rectangle [id:dp6426467945364456] 
\draw  [fill={rgb, 255:red, 230; green, 230; blue, 230 }  ,fill opacity=1 ] (314,289.56) -- (384,289.56) -- (384,316.75) -- (314,316.75) -- cycle ;
%Shape: Rectangle [id:dp6753744028699336] 
\draw  [fill={rgb, 255:red, 230; green, 230; blue, 230 }  ,fill opacity=1 ] (410,289.56) -- (480,289.56) -- (480,316.75) -- (410,316.75) -- cycle ;
%Shape: Rectangle [id:dp22653105184896782] 
\draw  [fill={rgb, 255:red, 230; green, 230; blue, 230 }  ,fill opacity=1 ] (506,289.56) -- (576,289.56) -- (576,316.75) -- (506,316.75) -- cycle ;
%Shape: Rectangle [id:dp6101363939979676] 
\draw  [fill={rgb, 255:red, 230; green, 230; blue, 230 }  ,fill opacity=1 ] (27,289.56) -- (97,289.56) -- (97,316.75) -- (27,316.75) -- cycle ;
%Straight Lines [id:da8562658424352227] 
\draw    (506,304.06) -- (484,304.06) ;
\draw [shift={(482,304.06)}, rotate = 360] [color={rgb, 255:red, 0; green, 0; blue, 0 }  ][line width=0.75]    (10.93,-3.29) .. controls (6.95,-1.4) and (3.31,-0.3) .. (0,0) .. controls (3.31,0.3) and (6.95,1.4) .. (10.93,3.29)   ;
%Straight Lines [id:da08741156076828793] 
\draw    (409,304.06) -- (387,304.06) ;
\draw [shift={(385,304.06)}, rotate = 360] [color={rgb, 255:red, 0; green, 0; blue, 0 }  ][line width=0.75]    (10.93,-3.29) .. controls (6.95,-1.4) and (3.31,-0.3) .. (0,0) .. controls (3.31,0.3) and (6.95,1.4) .. (10.93,3.29)   ;
%Straight Lines [id:da8638146000270468] 
\draw    (314,304.06) -- (292,304.06) ;
\draw [shift={(290,304.06)}, rotate = 360] [color={rgb, 255:red, 0; green, 0; blue, 0 }  ][line width=0.75]    (10.93,-3.29) .. controls (6.95,-1.4) and (3.31,-0.3) .. (0,0) .. controls (3.31,0.3) and (6.95,1.4) .. (10.93,3.29)   ;
%Straight Lines [id:da10427520515040589] 
\draw    (217,304.06) -- (195,304.06) ;
\draw [shift={(193,304.06)}, rotate = 360] [color={rgb, 255:red, 0; green, 0; blue, 0 }  ][line width=0.75]    (10.93,-3.29) .. controls (6.95,-1.4) and (3.31,-0.3) .. (0,0) .. controls (3.31,0.3) and (6.95,1.4) .. (10.93,3.29)   ;
%Straight Lines [id:da16233848692413888] 
\draw    (122,304.06) -- (100,304.06) ;
\draw [shift={(98,304.06)}, rotate = 360] [color={rgb, 255:red, 0; green, 0; blue, 0 }  ][line width=0.75]    (10.93,-3.29) .. controls (6.95,-1.4) and (3.31,-0.3) .. (0,0) .. controls (3.31,0.3) and (6.95,1.4) .. (10.93,3.29)   ;
%Shape: Rectangle [id:dp4358706438406543] 
\draw  [dash pattern={on 4.5pt off 4.5pt}] (498.83,283.67) -- (583,283.67) -- (583,353) -- (498.83,353) -- cycle ;
%Shape: Rectangle [id:dp9819663403360148] 
\draw  [fill={rgb, 255:red, 255; green, 200; blue, 200 }  ,fill opacity=1 ] (719,288) -- (789,288) -- (789,347.33) -- (719,347.33) -- cycle ;
%Shape: Rectangle [id:dp33824789321783477] 
\draw  [fill={rgb, 255:red, 255; green, 200; blue, 200 }  ,fill opacity=1 ] (815,288) -- (885,288) -- (885,347.33) -- (815,347.33) -- cycle ;
%Shape: Rectangle [id:dp6323144307115476] 
\draw  [fill={rgb, 255:red, 255; green, 200; blue, 200 }  ,fill opacity=1 ] (624,288) -- (694,288) -- (694,347.33) -- (624,347.33) -- cycle ;
%Straight Lines [id:da9506444597254975] 
\draw    (694,319) -- (716,319) ;
\draw [shift={(718,319)}, rotate = 180] [color={rgb, 255:red, 0; green, 0; blue, 0 }  ][line width=0.75]    (10.93,-3.29) .. controls (6.95,-1.4) and (3.31,-0.3) .. (0,0) .. controls (3.31,0.3) and (6.95,1.4) .. (10.93,3.29)   ;
%Straight Lines [id:da9938463777919722] 
\draw    (791,319) -- (813,319) ;
\draw [shift={(815,319)}, rotate = 180] [color={rgb, 255:red, 0; green, 0; blue, 0 }  ][line width=0.75]    (10.93,-3.29) .. controls (6.95,-1.4) and (3.31,-0.3) .. (0,0) .. controls (3.31,0.3) and (6.95,1.4) .. (10.93,3.29)   ;
%Straight Lines [id:da31955854007301676] 
\draw    (660.33,402) -- (660.33,356) ;
\draw [shift={(660.33,354)}, rotate = 450] [color={rgb, 255:red, 0; green, 0; blue, 0 }  ][line width=0.75]    (10.93,-3.29) .. controls (6.95,-1.4) and (3.31,-0.3) .. (0,0) .. controls (3.31,0.3) and (6.95,1.4) .. (10.93,3.29)   ;
%Straight Lines [id:da09255937714817297] 
\draw    (757,402) -- (757,356) ;
\draw [shift={(757,354)}, rotate = 450] [color={rgb, 255:red, 0; green, 0; blue, 0 }  ][line width=0.75]    (10.93,-3.29) .. controls (6.95,-1.4) and (3.31,-0.3) .. (0,0) .. controls (3.31,0.3) and (6.95,1.4) .. (10.93,3.29)   ;
%Straight Lines [id:da9795870337496946] 
\draw    (848,401) -- (848,355) ;
\draw [shift={(848,353)}, rotate = 450] [color={rgb, 255:red, 0; green, 0; blue, 0 }  ][line width=0.75]    (10.93,-3.29) .. controls (6.95,-1.4) and (3.31,-0.3) .. (0,0) .. controls (3.31,0.3) and (6.95,1.4) .. (10.93,3.29)   ;
%Curve Lines [id:da4726931354327524] 
\draw    (585.83,316.07) .. controls (625.23,295.98) and (583.3,366.91) .. (620.26,349.55) ;
\draw [shift={(622,348.7)}, rotate = 512.99] [color={rgb, 255:red, 0; green, 0; blue, 0 }  ][line width=0.75]    (10.93,-3.29) .. controls (6.95,-1.4) and (3.31,-0.3) .. (0,0) .. controls (3.31,0.3) and (6.95,1.4) .. (10.93,3.29)   ;
%Shape: Rectangle [id:dp573190615817307] 
\draw  [fill={rgb, 255:red, 255; green, 255; blue, 255 }  ,fill opacity=1 ] (251.33,156.67) -- (467.33,156.67) -- (467.33,196.67) -- (251.33,196.67) -- cycle ;
%Straight Lines [id:da1800551291558734] 
\draw    (658.67,259) -- (658.67,213) ;
\draw [shift={(658.67,211)}, rotate = 450] [color={rgb, 255:red, 0; green, 0; blue, 0 }  ][line width=0.75]    (10.93,-3.29) .. controls (6.95,-1.4) and (3.31,-0.3) .. (0,0) .. controls (3.31,0.3) and (6.95,1.4) .. (10.93,3.29)   ;
%Shape: Rectangle [id:dp017660385315100058] 
\draw  [fill={rgb, 255:red, 255; green, 231; blue, 191 }  ,fill opacity=1 ] (624,259) -- (696,259) -- (696,281.33) -- (624,281.33) -- cycle ;
%Straight Lines [id:da6816118834155438] 
\draw    (60.5,285.25) -- (310.08,211.5) ;
\draw [shift={(312,210.94)}, rotate = 523.54] [color={rgb, 255:red, 0; green, 0; blue, 0 }  ][line width=0.75]    (10.93,-3.29) .. controls (6.95,-1.4) and (3.31,-0.3) .. (0,0) .. controls (3.31,0.3) and (6.95,1.4) .. (10.93,3.29)   ;
%Straight Lines [id:da07351465426359427] 
\draw    (155.83,284.8) -- (334.15,211.7) ;
\draw [shift={(336,210.94)}, rotate = 517.71] [color={rgb, 255:red, 0; green, 0; blue, 0 }  ][line width=0.75]    (10.93,-3.29) .. controls (6.95,-1.4) and (3.31,-0.3) .. (0,0) .. controls (3.31,0.3) and (6.95,1.4) .. (10.93,3.29)   ;
%Straight Lines [id:da7003797871444293] 
\draw    (251.83,283.89) -- (346.9,212.14) ;
\draw [shift={(348.5,210.94)}, rotate = 502.96] [color={rgb, 255:red, 0; green, 0; blue, 0 }  ][line width=0.75]    (10.93,-3.29) .. controls (6.95,-1.4) and (3.31,-0.3) .. (0,0) .. controls (3.31,0.3) and (6.95,1.4) .. (10.93,3.29)   ;
%Straight Lines [id:da04846794396624077] 
\draw    (348.5,283.89) -- (359.69,212.91) ;
\draw [shift={(360,210.94)}, rotate = 458.96] [color={rgb, 255:red, 0; green, 0; blue, 0 }  ][line width=0.75]    (10.93,-3.29) .. controls (6.95,-1.4) and (3.31,-0.3) .. (0,0) .. controls (3.31,0.3) and (6.95,1.4) .. (10.93,3.29)   ;
%Straight Lines [id:da07677692002640724] 
\draw    (445.17,284.8) -- (372.58,212.35) ;
\draw [shift={(371.17,210.94)}, rotate = 404.95] [color={rgb, 255:red, 0; green, 0; blue, 0 }  ][line width=0.75]    (10.93,-3.29) .. controls (6.95,-1.4) and (3.31,-0.3) .. (0,0) .. controls (3.31,0.3) and (6.95,1.4) .. (10.93,3.29)   ;
%Straight Lines [id:da7782552459714387] 
\draw    (538.5,285.25) -- (385.8,211.8) ;
\draw [shift={(384,210.94)}, rotate = 385.69] [color={rgb, 255:red, 0; green, 0; blue, 0 }  ][line width=0.75]    (10.93,-3.29) .. controls (6.95,-1.4) and (3.31,-0.3) .. (0,0) .. controls (3.31,0.3) and (6.95,1.4) .. (10.93,3.29)   ;
%Curve Lines [id:da1838475336218326] 
\draw [color={rgb, 255:red, 246; green, 155; blue, 5 }  ,draw opacity=1 ]   (467.83,180.67) .. controls (507.43,150.97) and (571.86,257.83) .. (622.47,263.85) ;
\draw [shift={(624,264)}, rotate = 184.5] [color={rgb, 255:red, 246; green, 155; blue, 5 }  ,draw opacity=1 ][line width=0.75]    (10.93,-3.29) .. controls (6.95,-1.4) and (3.31,-0.3) .. (0,0) .. controls (3.31,0.3) and (6.95,1.4) .. (10.93,3.29)   ;
%Straight Lines [id:da18797641002683707] 
\draw [color={rgb, 255:red, 0; green, 0; blue, 0 }  ,draw opacity=1 ]   (624,308.82) -- (469.38,181.94) ;
\draw [shift={(467.83,180.67)}, rotate = 399.37] [color={rgb, 255:red, 0; green, 0; blue, 0 }  ,draw opacity=1 ][line width=0.75]    (10.93,-3.29) .. controls (6.95,-1.4) and (3.31,-0.3) .. (0,0) .. controls (3.31,0.3) and (6.95,1.4) .. (10.93,3.29)   ;
%Shape: Rectangle [id:dp23976218799558513] 
\draw  [fill={rgb, 255:red, 255; green, 231; blue, 191 }  ,fill opacity=1 ] (720,260) -- (792,260) -- (792,281.33) -- (720,281.33) -- cycle ;
%Shape: Rectangle [id:dp9150697308665792] 
\draw  [fill={rgb, 255:red, 255; green, 231; blue, 191 }  ,fill opacity=1 ] (816,261) -- (888,261) -- (888,281.33) -- (816,281.33) -- cycle ;
%Straight Lines [id:da8949651191899977] 
\draw    (756.67,260) -- (756.67,214) ;
\draw [shift={(756.67,212)}, rotate = 450] [color={rgb, 255:red, 0; green, 0; blue, 0 }  ][line width=0.75]    (10.93,-3.29) .. controls (6.95,-1.4) and (3.31,-0.3) .. (0,0) .. controls (3.31,0.3) and (6.95,1.4) .. (10.93,3.29)   ;
%Straight Lines [id:da26766392260011895] 
\draw    (851.67,261) -- (851.67,215) ;
\draw [shift={(851.67,213)}, rotate = 450] [color={rgb, 255:red, 0; green, 0; blue, 0 }  ][line width=0.75]    (10.93,-3.29) .. controls (6.95,-1.4) and (3.31,-0.3) .. (0,0) .. controls (3.31,0.3) and (6.95,1.4) .. (10.93,3.29)   ;
%Straight Lines [id:da3781874946177224] 
\draw    (696,175) -- (718,175) ;
\draw [shift={(720,175)}, rotate = 180] [color={rgb, 255:red, 0; green, 0; blue, 0 }  ][line width=0.75]    (10.93,-3.29) .. controls (6.95,-1.4) and (3.31,-0.3) .. (0,0) .. controls (3.31,0.3) and (6.95,1.4) .. (10.93,3.29)   ;
%Straight Lines [id:da7095061905887059] 
\draw    (792,174) -- (814,174) ;
\draw [shift={(816,174)}, rotate = 180] [color={rgb, 255:red, 0; green, 0; blue, 0 }  ][line width=0.75]    (10.93,-3.29) .. controls (6.95,-1.4) and (3.31,-0.3) .. (0,0) .. controls (3.31,0.3) and (6.95,1.4) .. (10.93,3.29)   ;
%Straight Lines [id:da6612246196817655] 
\draw    (659,135) -- (659,113) ;
\draw [shift={(659,111)}, rotate = 450] [color={rgb, 255:red, 0; green, 0; blue, 0 }  ][line width=0.75]    (10.93,-3.29) .. controls (6.95,-1.4) and (3.31,-0.3) .. (0,0) .. controls (3.31,0.3) and (6.95,1.4) .. (10.93,3.29)   ;
%Straight Lines [id:da9338273912038508] 
\draw    (756,135.5) -- (756,113.5) ;
\draw [shift={(756,111.5)}, rotate = 450] [color={rgb, 255:red, 0; green, 0; blue, 0 }  ][line width=0.75]    (10.93,-3.29) .. controls (6.95,-1.4) and (3.31,-0.3) .. (0,0) .. controls (3.31,0.3) and (6.95,1.4) .. (10.93,3.29)   ;
%Straight Lines [id:da0015725499467724724] 
\draw    (851.5,140) -- (851.5,118) ;
\draw [shift={(851.5,116)}, rotate = 450] [color={rgb, 255:red, 0; green, 0; blue, 0 }  ][line width=0.75]    (10.93,-3.29) .. controls (6.95,-1.4) and (3.31,-0.3) .. (0,0) .. controls (3.31,0.3) and (6.95,1.4) .. (10.93,3.29)   ;
%Shape: Rectangle [id:dp46109286792869986] 
\draw   (718,69) -- (788,69) -- (788,102.5) -- (718,102.5) -- cycle ;

%Shape: Rectangle [id:dp16695737613580652] 
\draw   (814,69) -- (884,69) -- (884,102.5) -- (814,102.5) -- cycle ;

%Shape: Rectangle [id:dp02151681285353635] 
\draw   (624,69) -- (694,69) -- (694,102.5) -- (624,102.5) -- cycle ;

%Curve Lines [id:da5825212906474626] 
\draw    (585.83,316.07) .. controls (625.63,295.78) and (557.19,177.35) .. (623,168.13) ;
\draw [shift={(624,168)}, rotate = 533.11] [color={rgb, 255:red, 0; green, 0; blue, 0 }  ][line width=0.75]    (10.93,-3.29) .. controls (6.95,-1.4) and (3.31,-0.3) .. (0,0) .. controls (3.31,0.3) and (6.95,1.4) .. (10.93,3.29)   ;
%Shape: Rectangle [id:dp5688982651250001] 
\draw  [fill={rgb, 255:red, 255; green, 200; blue, 200 }  ,fill opacity=1 ] (624,143) -- (694,143) -- (694,202.33) -- (624,202.33) -- cycle ;
%Shape: Rectangle [id:dp05692065112254996] 
\draw  [fill={rgb, 255:red, 255; green, 200; blue, 200 }  ,fill opacity=1 ] (722,144) -- (792,144) -- (792,203.33) -- (722,203.33) -- cycle ;
%Shape: Rectangle [id:dp4656585586178468] 
\draw  [fill={rgb, 255:red, 255; green, 200; blue, 200 }  ,fill opacity=1 ] (818,144) -- (888,144) -- (888,203.33) -- (818,203.33) -- cycle ;

% Text Node
\draw (263,534) node [anchor=north west][inner sep=0.75pt]  [font=\Large,color={rgb, 255:red, 26; green, 66; blue, 114 }  ,opacity=1 ] [align=left] {Avaliação};
% Text Node
\draw (359.33,176.67) node   [align=left] {\begin{minipage}[lt]{145.20137142857146pt}\setlength\topsep{0pt}
\begin{center}
Atenção
\end{center}

\end{minipage}};
% Text Node
\draw (891.5,490) node [anchor=north west][inner sep=0.75pt]   [align=left] {...};
% Text Node
\draw (891,310) node [anchor=north west][inner sep=0.75pt]   [align=left] {...};
% Text Node
\draw (892,168) node [anchor=north west][inner sep=0.75pt]   [align=left] {...};
% Text Node
\draw (891,81) node [anchor=north west][inner sep=0.75pt]   [align=left] {...};
% Text Node
\draw (719,531) node [anchor=north west][inner sep=0.75pt]  [font=\Large,color={rgb, 255:red, 26; green, 66; blue, 114 }  ,opacity=1 ] [align=left] {\textcolor[rgb]{0,0,0}{Título}};
% Text Node
\draw (650,262) node [anchor=north west][inner sep=0.75pt]   [align=left] {$\displaystyle c_{1}$};
% Text Node
\draw (747,263) node [anchor=north west][inner sep=0.75pt]   [align=left] {$\displaystyle c_{2}$};
% Text Node
\draw (841,263) node [anchor=north west][inner sep=0.75pt]   [align=left] {$\displaystyle c_{3}$};
% Text Node
\draw (468,403) node [anchor=north west][inner sep=0.75pt]   [align=left] {Emb};
% Text Node
\draw (659.2,85.75) node   [align=left] {\begin{minipage}[lt]{47.05600000000003pt}\setlength\topsep{0pt}
\begin{center}
Veio
\end{center}

\end{minipage}};
% Text Node
\draw (849.2,85.75) node   [align=left] {\begin{minipage}[lt]{47.05600000000003pt}\setlength\topsep{0pt}
\begin{center}
defeito
\end{center}

\end{minipage}};
% Text Node
\draw (753.2,85.75) node   [align=left] {\begin{minipage}[lt]{47.05600000000003pt}\setlength\topsep{0pt}
\begin{center}
com
\end{center}

\end{minipage}};
% Text Node
\draw (661.2,497) node   [align=left] {\begin{minipage}[lt]{47.05600000000001pt}\setlength\topsep{0pt}
\begin{center}
[bos]
\end{center}

\end{minipage}};
% Text Node
\draw (851.2,497) node   [align=left] {\begin{minipage}[lt]{47.05600000000001pt}\setlength\topsep{0pt}
\begin{center}
com
\end{center}

\end{minipage}};
% Text Node
\draw (755.2,497) node   [align=left] {\begin{minipage}[lt]{47.05600000000001pt}\setlength\topsep{0pt}
\begin{center}
Veio
\end{center}

\end{minipage}};
% Text Node
\draw (63.2,497) node   [align=left] {\begin{minipage}[lt]{47.05600000000001pt}\setlength\topsep{0pt}
\begin{center}
[bos]
\end{center}

\end{minipage}};
% Text Node
\draw (541.2,497) node   [align=left] {\begin{minipage}[lt]{47.05600000000001pt}\setlength\topsep{0pt}
\begin{center}
[eos]
\end{center}

\end{minipage}};
% Text Node
\draw (445.2,497) node   [align=left] {\begin{minipage}[lt]{47.05600000000001pt}\setlength\topsep{0pt}
\begin{center}
liga
\end{center}

\end{minipage}};
% Text Node
\draw (349.2,497) node   [align=left] {\begin{minipage}[lt]{47.05600000000001pt}\setlength\topsep{0pt}
\begin{center}
não
\end{center}

\end{minipage}};
% Text Node
\draw (253.2,497) node   [align=left] {\begin{minipage}[lt]{47.05600000000001pt}\setlength\topsep{0pt}
\begin{center}
tv
\end{center}

\end{minipage}};
% Text Node
\draw (157.2,497) node   [align=left] {\begin{minipage}[lt]{47.05600000000001pt}\setlength\topsep{0pt}
\begin{center}
A
\end{center}

\end{minipage}};


\end{tikzpicture}

\label{lstm_fig}
\caption{Diagrama do modelo encoder-decoder BiLSTM com camada de atenção escolhido. Os tokens especiais \texttt{[bos]} e \texttt{[eos]} delimitam respectivamente o início e fim das sequências de texto.}
\centering
\end{figure}

O mecanismos de atenção adotado para o modelo foi o mecanismo de atenção global proposto por \textcite{luong2015effective} com escores de atenção obtidos a partir do produto interno dos \textit{hidden states}. Graças à segunda LSTM do decoder, o modelo também é capaz de utilizar a informação dos vetores de contexto dos tokens passados.


\subsection{BERT-CLS e BERT-MASK}

Neste trabalho iremos experimentar com duas variantes de soluções baseadas no BERT (representado na Figura \ref{bert_fig}) para geração de textos de maneira autoregressiva. A primeira variante consiste em utilizar o campo de classificação (token \texttt{[CLS]}) de um modelo BERT pré-treinado para a língua portuguesa (\textcite{souza2020bertimbau}) para prever o próximo token de uma sequência.

A camada final associada ao token \texttt{[CLS]} do BERT é pré-treinada com a tarefa de \textit{Next Sentence Prediction} (NSP), que consiste em tentar adivinhar se a segunda sentença fornecida (separada pelo token \texttt{[SEP]}) vem depois da primeira sentença em um texto corrido. Para o problema em questão, podemos descartar a camada densa de classificação binária usada para o objetivo de NSP e criar uma nova camada densa com a mesma dimensão final do vocabulário de saída, e utilizarmos essa nova componente para prever o próximo token do título de maneira autoregressiva. Chamaremos esse modelo de BERT-CLS.

\begin{figure}[h]
	

\tikzset{every picture/.style={line width=0.75pt}} %set default line width to 0.75pt        

\begin{tikzpicture}[x=0.75pt,y=0.75pt,yscale=-0.65,xscale=0.65]
%uncomment if require: \path (0,627); %set diagram left start at 0, and has height of 627


%uncomment if require: \path (0,627); %set diagram left start at 0, and has height of 627

%Shape: Rectangle [id:dp937218640055631] 
\draw  [fill={rgb, 255:red, 204; green, 239; blue, 255 }  ,fill opacity=1 ] (175,513) -- (245,513) -- (245,548) -- (175,548) -- cycle ;

%Shape: Rectangle [id:dp4687154989872446] 
\draw  [fill={rgb, 255:red, 204; green, 239; blue, 255 }  ,fill opacity=1 ] (280,513) -- (350,513) -- (350,548) -- (280,548) -- cycle ;

%Shape: Rectangle [id:dp886569090318859] 
\draw  [fill={rgb, 255:red, 204; green, 239; blue, 255 }  ,fill opacity=1 ] (385,513) -- (455,513) -- (455,548) -- (385,548) -- cycle ;

%Shape: Rectangle [id:dp34502295864245025] 
\draw  [fill={rgb, 255:red, 204; green, 239; blue, 255 }  ,fill opacity=1 ] (490,513) -- (560,513) -- (560,548) -- (490,548) -- cycle ;

%Shape: Rectangle [id:dp5908363712480504] 
\draw  [fill={rgb, 255:red, 255; green, 255; blue, 255 }  ,fill opacity=1 ] (595,513) -- (665,513) -- (665,548) -- (595,548) -- cycle ;

%Shape: Rectangle [id:dp8281428094239127] 
\draw  [fill={rgb, 255:red, 255; green, 255; blue, 255 }  ,fill opacity=1 ] (70,513) -- (140,513) -- (140,548) -- (70,548) -- cycle ;

%Shape: Rectangle [id:dp49591920725844907] 
\draw  [fill={rgb, 255:red, 204; green, 255; blue, 218 }  ,fill opacity=1 ] (700,513) -- (770,513) -- (770,548) -- (700,548) -- cycle ;

%Shape: Rectangle [id:dp5379403291623639] 
\draw  [fill={rgb, 255:red, 204; green, 255; blue, 218 }  ,fill opacity=1 ] (805,513) -- (875,513) -- (875,548) -- (805,548) -- cycle ;

%Straight Lines [id:da9759381815964863] 
\draw    (104,513) -- (104,467) ;
\draw [shift={(104,465)}, rotate = 450] [color={rgb, 255:red, 0; green, 0; blue, 0 }  ][line width=0.75]    (10.93,-3.29) .. controls (6.95,-1.4) and (3.31,-0.3) .. (0,0) .. controls (3.31,0.3) and (6.95,1.4) .. (10.93,3.29)   ;
%Shape: Rectangle [id:dp20256526671871877] 
\draw  [fill={rgb, 255:red, 255; green, 255; blue, 255 }  ,fill opacity=1 ] (910,513) -- (980,513) -- (980,548) -- (910,548) -- cycle ;

%Straight Lines [id:da4789618172504544] 
\draw    (210,513) -- (210,467) ;
\draw [shift={(210,465)}, rotate = 450] [color={rgb, 255:red, 0; green, 0; blue, 0 }  ][line width=0.75]    (10.93,-3.29) .. controls (6.95,-1.4) and (3.31,-0.3) .. (0,0) .. controls (3.31,0.3) and (6.95,1.4) .. (10.93,3.29)   ;
%Straight Lines [id:da6278500111729917] 
\draw    (315,513) -- (315,467) ;
\draw [shift={(315,465)}, rotate = 450] [color={rgb, 255:red, 0; green, 0; blue, 0 }  ][line width=0.75]    (10.93,-3.29) .. controls (6.95,-1.4) and (3.31,-0.3) .. (0,0) .. controls (3.31,0.3) and (6.95,1.4) .. (10.93,3.29)   ;
%Straight Lines [id:da6349608451576887] 
\draw    (420,513) -- (420,467) ;
\draw [shift={(420,465)}, rotate = 450] [color={rgb, 255:red, 0; green, 0; blue, 0 }  ][line width=0.75]    (10.93,-3.29) .. controls (6.95,-1.4) and (3.31,-0.3) .. (0,0) .. controls (3.31,0.3) and (6.95,1.4) .. (10.93,3.29)   ;
%Straight Lines [id:da058074609133208055] 
\draw    (525,513) -- (525,467) ;
\draw [shift={(525,465)}, rotate = 450] [color={rgb, 255:red, 0; green, 0; blue, 0 }  ][line width=0.75]    (10.93,-3.29) .. controls (6.95,-1.4) and (3.31,-0.3) .. (0,0) .. controls (3.31,0.3) and (6.95,1.4) .. (10.93,3.29)   ;
%Straight Lines [id:da8759984181623635] 
\draw    (630,513) -- (630,467) ;
\draw [shift={(630,465)}, rotate = 450] [color={rgb, 255:red, 0; green, 0; blue, 0 }  ][line width=0.75]    (10.93,-3.29) .. controls (6.95,-1.4) and (3.31,-0.3) .. (0,0) .. controls (3.31,0.3) and (6.95,1.4) .. (10.93,3.29)   ;
%Straight Lines [id:da7405270066960039] 
\draw    (735,513) -- (735,467) ;
\draw [shift={(735,465)}, rotate = 450] [color={rgb, 255:red, 0; green, 0; blue, 0 }  ][line width=0.75]    (10.93,-3.29) .. controls (6.95,-1.4) and (3.31,-0.3) .. (0,0) .. controls (3.31,0.3) and (6.95,1.4) .. (10.93,3.29)   ;
%Straight Lines [id:da4215840073406023] 
\draw    (840,513) -- (840,467) ;
\draw [shift={(840,465)}, rotate = 450] [color={rgb, 255:red, 0; green, 0; blue, 0 }  ][line width=0.75]    (10.93,-3.29) .. controls (6.95,-1.4) and (3.31,-0.3) .. (0,0) .. controls (3.31,0.3) and (6.95,1.4) .. (10.93,3.29)   ;
%Straight Lines [id:da14847506547236344] 
\draw    (945,513) -- (945,467) ;
\draw [shift={(945,465)}, rotate = 450] [color={rgb, 255:red, 0; green, 0; blue, 0 }  ][line width=0.75]    (10.93,-3.29) .. controls (6.95,-1.4) and (3.31,-0.3) .. (0,0) .. controls (3.31,0.3) and (6.95,1.4) .. (10.93,3.29)   ;
%Rounded Rect [id:dp8607721076656145] 
\draw  [fill={rgb, 255:red, 255; green, 237; blue, 237 }  ,fill opacity=1 ] (70,364) .. controls (70,352.4) and (79.4,343) .. (91,343) -- (959,343) .. controls (970.6,343) and (980,352.4) .. (980,364) -- (980,427) .. controls (980,438.6) and (970.6,448) .. (959,448) -- (91,448) .. controls (79.4,448) and (70,438.6) .. (70,427) -- cycle ;

%Straight Lines [id:da96555737981149] 
\draw    (105,321) -- (105,275) ;
\draw [shift={(105,273)}, rotate = 450] [color={rgb, 255:red, 0; green, 0; blue, 0 }  ][line width=0.75]    (10.93,-3.29) .. controls (6.95,-1.4) and (3.31,-0.3) .. (0,0) .. controls (3.31,0.3) and (6.95,1.4) .. (10.93,3.29)   ;
%Straight Lines [id:da5727342275811496] 
\draw    (211,321) -- (211,275) ;
\draw [shift={(211,273)}, rotate = 450] [color={rgb, 255:red, 0; green, 0; blue, 0 }  ][line width=0.75]    (10.93,-3.29) .. controls (6.95,-1.4) and (3.31,-0.3) .. (0,0) .. controls (3.31,0.3) and (6.95,1.4) .. (10.93,3.29)   ;
%Straight Lines [id:da8041766474175829] 
\draw    (316,321) -- (316,275) ;
\draw [shift={(316,273)}, rotate = 450] [color={rgb, 255:red, 0; green, 0; blue, 0 }  ][line width=0.75]    (10.93,-3.29) .. controls (6.95,-1.4) and (3.31,-0.3) .. (0,0) .. controls (3.31,0.3) and (6.95,1.4) .. (10.93,3.29)   ;
%Straight Lines [id:da395735529278592] 
\draw    (421,321) -- (421,275) ;
\draw [shift={(421,273)}, rotate = 450] [color={rgb, 255:red, 0; green, 0; blue, 0 }  ][line width=0.75]    (10.93,-3.29) .. controls (6.95,-1.4) and (3.31,-0.3) .. (0,0) .. controls (3.31,0.3) and (6.95,1.4) .. (10.93,3.29)   ;
%Straight Lines [id:da7073063987813077] 
\draw    (526,321) -- (526,275) ;
\draw [shift={(526,273)}, rotate = 450] [color={rgb, 255:red, 0; green, 0; blue, 0 }  ][line width=0.75]    (10.93,-3.29) .. controls (6.95,-1.4) and (3.31,-0.3) .. (0,0) .. controls (3.31,0.3) and (6.95,1.4) .. (10.93,3.29)   ;
%Straight Lines [id:da9003539337446975] 
\draw    (631,321) -- (631,275) ;
\draw [shift={(631,273)}, rotate = 450] [color={rgb, 255:red, 0; green, 0; blue, 0 }  ][line width=0.75]    (10.93,-3.29) .. controls (6.95,-1.4) and (3.31,-0.3) .. (0,0) .. controls (3.31,0.3) and (6.95,1.4) .. (10.93,3.29)   ;
%Straight Lines [id:da6834772814034134] 
\draw    (736,321) -- (736,275) ;
\draw [shift={(736,273)}, rotate = 450] [color={rgb, 255:red, 0; green, 0; blue, 0 }  ][line width=0.75]    (10.93,-3.29) .. controls (6.95,-1.4) and (3.31,-0.3) .. (0,0) .. controls (3.31,0.3) and (6.95,1.4) .. (10.93,3.29)   ;
%Straight Lines [id:da677706110283482] 
\draw    (841,321) -- (841,275) ;
\draw [shift={(841,273)}, rotate = 450] [color={rgb, 255:red, 0; green, 0; blue, 0 }  ][line width=0.75]    (10.93,-3.29) .. controls (6.95,-1.4) and (3.31,-0.3) .. (0,0) .. controls (3.31,0.3) and (6.95,1.4) .. (10.93,3.29)   ;
%Straight Lines [id:da6874911396172543] 
\draw    (946,321) -- (946,275) ;
\draw [shift={(946,273)}, rotate = 450] [color={rgb, 255:red, 0; green, 0; blue, 0 }  ][line width=0.75]    (10.93,-3.29) .. controls (6.95,-1.4) and (3.31,-0.3) .. (0,0) .. controls (3.31,0.3) and (6.95,1.4) .. (10.93,3.29)   ;
%Shape: Rectangle [id:dp6075434400048438] 
\draw  [fill={rgb, 255:red, 255; green, 225; blue, 176 }  ,fill opacity=1 ] (175,198) -- (245,198) -- (245,254.6) -- (175,254.6) -- cycle ;

%Shape: Rectangle [id:dp9390214311090217] 
\draw  [fill={rgb, 255:red, 255; green, 225; blue, 176 }  ,fill opacity=1 ] (280,198) -- (350,198) -- (350,254.6) -- (280,254.6) -- cycle ;

%Shape: Rectangle [id:dp7897665542900947] 
\draw  [fill={rgb, 255:red, 255; green, 225; blue, 176 }  ,fill opacity=1 ] (385,198) -- (455,198) -- (455,254.6) -- (385,254.6) -- cycle ;

%Shape: Rectangle [id:dp20403620418198987] 
\draw  [fill={rgb, 255:red, 255; green, 225; blue, 176 }  ,fill opacity=1 ] (490,198) -- (560,198) -- (560,254.6) -- (490,254.6) -- cycle ;

%Shape: Rectangle [id:dp9659699874280345] 
\draw  [fill={rgb, 255:red, 255; green, 225; blue, 176 }  ,fill opacity=1 ] (595,198) -- (665,198) -- (665,254.6) -- (595,254.6) -- cycle ;

%Shape: Rectangle [id:dp7137845675985124] 
\draw  [fill={rgb, 255:red, 255; green, 186; blue, 187 }  ,fill opacity=1 ] (70,198) -- (140,198) -- (140,254.6) -- (70,254.6) -- cycle ;

%Shape: Rectangle [id:dp7121654166244109] 
\draw  [fill={rgb, 255:red, 255; green, 225; blue, 176 }  ,fill opacity=1 ] (700,198) -- (770,198) -- (770,254.6) -- (700,254.6) -- cycle ;

%Shape: Rectangle [id:dp2848572824465041] 
\draw  [fill={rgb, 255:red, 255; green, 225; blue, 176 }  ,fill opacity=1 ] (805,198) -- (875,198) -- (875,254.6) -- (805,254.6) -- cycle ;

%Shape: Rectangle [id:dp1334921794682138] 
\draw  [fill={rgb, 255:red, 255; green, 225; blue, 176 }  ,fill opacity=1 ] (910,198) -- (980,198) -- (980,254.6) -- (910,254.6) -- cycle ;

%Straight Lines [id:da8237337419143655] 
\draw    (105,187) -- (105,141) ;
\draw [shift={(105,139)}, rotate = 450] [color={rgb, 255:red, 0; green, 0; blue, 0 }  ][line width=0.75]    (10.93,-3.29) .. controls (6.95,-1.4) and (3.31,-0.3) .. (0,0) .. controls (3.31,0.3) and (6.95,1.4) .. (10.93,3.29)   ;
%Straight Lines [id:da35863700512517216] 
\draw    (211,187) -- (211,141) ;
\draw [shift={(211,139)}, rotate = 450] [color={rgb, 255:red, 0; green, 0; blue, 0 }  ][line width=0.75]    (10.93,-3.29) .. controls (6.95,-1.4) and (3.31,-0.3) .. (0,0) .. controls (3.31,0.3) and (6.95,1.4) .. (10.93,3.29)   ;
%Straight Lines [id:da5033652828776451] 
\draw    (316,187) -- (316,141) ;
\draw [shift={(316,139)}, rotate = 450] [color={rgb, 255:red, 0; green, 0; blue, 0 }  ][line width=0.75]    (10.93,-3.29) .. controls (6.95,-1.4) and (3.31,-0.3) .. (0,0) .. controls (3.31,0.3) and (6.95,1.4) .. (10.93,3.29)   ;
%Straight Lines [id:da3835860160796918] 
\draw    (421,187) -- (421,141) ;
\draw [shift={(421,139)}, rotate = 450] [color={rgb, 255:red, 0; green, 0; blue, 0 }  ][line width=0.75]    (10.93,-3.29) .. controls (6.95,-1.4) and (3.31,-0.3) .. (0,0) .. controls (3.31,0.3) and (6.95,1.4) .. (10.93,3.29)   ;
%Straight Lines [id:da41902488920364855] 
\draw    (526,187) -- (526,141) ;
\draw [shift={(526,139)}, rotate = 450] [color={rgb, 255:red, 0; green, 0; blue, 0 }  ][line width=0.75]    (10.93,-3.29) .. controls (6.95,-1.4) and (3.31,-0.3) .. (0,0) .. controls (3.31,0.3) and (6.95,1.4) .. (10.93,3.29)   ;
%Straight Lines [id:da6962908186098935] 
\draw    (631,187) -- (631,141) ;
\draw [shift={(631,139)}, rotate = 450] [color={rgb, 255:red, 0; green, 0; blue, 0 }  ][line width=0.75]    (10.93,-3.29) .. controls (6.95,-1.4) and (3.31,-0.3) .. (0,0) .. controls (3.31,0.3) and (6.95,1.4) .. (10.93,3.29)   ;
%Straight Lines [id:da37173575450894525] 
\draw    (736,187) -- (736,141) ;
\draw [shift={(736,139)}, rotate = 450] [color={rgb, 255:red, 0; green, 0; blue, 0 }  ][line width=0.75]    (10.93,-3.29) .. controls (6.95,-1.4) and (3.31,-0.3) .. (0,0) .. controls (3.31,0.3) and (6.95,1.4) .. (10.93,3.29)   ;
%Straight Lines [id:da5692829474992189] 
\draw    (841,187) -- (841,141) ;
\draw [shift={(841,139)}, rotate = 450] [color={rgb, 255:red, 0; green, 0; blue, 0 }  ][line width=0.75]    (10.93,-3.29) .. controls (6.95,-1.4) and (3.31,-0.3) .. (0,0) .. controls (3.31,0.3) and (6.95,1.4) .. (10.93,3.29)   ;
%Straight Lines [id:da3706093679866933] 
\draw    (946,187) -- (946,141) ;
\draw [shift={(946,139)}, rotate = 450] [color={rgb, 255:red, 0; green, 0; blue, 0 }  ][line width=0.75]    (10.93,-3.29) .. controls (6.95,-1.4) and (3.31,-0.3) .. (0,0) .. controls (3.31,0.3) and (6.95,1.4) .. (10.93,3.29)   ;
%Shape: Rectangle [id:dp5236546342025983] 
\draw   (97,94) -- (113,94) -- (113,110) -- (97,110) -- cycle ;
%Shape: Rectangle [id:dp23554459933812644] 
\draw   (97,110) -- (113,110) -- (113,126) -- (97,126) -- cycle ;
%Shape: Rectangle [id:dp47720445708455683] 
\draw   (203,66) -- (219,66) -- (219,82) -- (203,82) -- cycle ;
%Shape: Rectangle [id:dp11348789690976147] 
\draw   (203,82) -- (219,82) -- (219,98) -- (203,98) -- cycle ;
%Shape: Rectangle [id:dp2989274466547829] 
\draw   (203,98) -- (219,98) -- (219,114) -- (203,114) -- cycle ;
%Shape: Rectangle [id:dp38220336303348335] 
\draw   (203,114) -- (219,114) -- (219,130) -- (203,130) -- cycle ;
%Shape: Rectangle [id:dp5015449138513748] 
\draw   (203,50) -- (219,50) -- (219,66) -- (203,66) -- cycle ;
%Shape: Rectangle [id:dp7901601617452054] 
\draw   (307,66) -- (323,66) -- (323,82) -- (307,82) -- cycle ;
%Shape: Rectangle [id:dp3898574911962034] 
\draw   (307,82) -- (323,82) -- (323,98) -- (307,98) -- cycle ;
%Shape: Rectangle [id:dp24946291294831058] 
\draw   (307,98) -- (323,98) -- (323,114) -- (307,114) -- cycle ;
%Shape: Rectangle [id:dp09151772199564978] 
\draw   (307,114) -- (323,114) -- (323,130) -- (307,130) -- cycle ;
%Shape: Rectangle [id:dp5389291799041205] 
\draw   (307,50) -- (323,50) -- (323,66) -- (307,66) -- cycle ;
%Shape: Rectangle [id:dp4411849161355179] 
\draw   (413,66) -- (429,66) -- (429,82) -- (413,82) -- cycle ;
%Shape: Rectangle [id:dp9414741395206954] 
\draw   (413,82) -- (429,82) -- (429,98) -- (413,98) -- cycle ;
%Shape: Rectangle [id:dp8738995037323045] 
\draw   (413,98) -- (429,98) -- (429,114) -- (413,114) -- cycle ;
%Shape: Rectangle [id:dp21734489177549032] 
\draw   (413,114) -- (429,114) -- (429,130) -- (413,130) -- cycle ;
%Shape: Rectangle [id:dp7406179111228264] 
\draw   (413,50) -- (429,50) -- (429,66) -- (413,66) -- cycle ;
%Shape: Rectangle [id:dp44998302387018785] 
\draw   (519,65) -- (535,65) -- (535,81) -- (519,81) -- cycle ;
%Shape: Rectangle [id:dp388003879087665] 
\draw   (519,81) -- (535,81) -- (535,97) -- (519,97) -- cycle ;
%Shape: Rectangle [id:dp46390585910681903] 
\draw   (519,97) -- (535,97) -- (535,113) -- (519,113) -- cycle ;
%Shape: Rectangle [id:dp8772574612035773] 
\draw   (519,113) -- (535,113) -- (535,129) -- (519,129) -- cycle ;
%Shape: Rectangle [id:dp8397238555646047] 
\draw   (519,49) -- (535,49) -- (535,65) -- (519,65) -- cycle ;
%Shape: Rectangle [id:dp6961896298597612] 
\draw   (623,65) -- (639,65) -- (639,81) -- (623,81) -- cycle ;
%Shape: Rectangle [id:dp11760166949090767] 
\draw   (623,81) -- (639,81) -- (639,97) -- (623,97) -- cycle ;
%Shape: Rectangle [id:dp3520243122407767] 
\draw   (623,97) -- (639,97) -- (639,113) -- (623,113) -- cycle ;
%Shape: Rectangle [id:dp9759570052212048] 
\draw   (623,113) -- (639,113) -- (639,129) -- (623,129) -- cycle ;
%Shape: Rectangle [id:dp09813722728317642] 
\draw   (623,49) -- (639,49) -- (639,65) -- (623,65) -- cycle ;
%Shape: Rectangle [id:dp9252046267854279] 
\draw   (727,65) -- (743,65) -- (743,81) -- (727,81) -- cycle ;
%Shape: Rectangle [id:dp8730078373170074] 
\draw   (727,81) -- (743,81) -- (743,97) -- (727,97) -- cycle ;
%Shape: Rectangle [id:dp9089462585251233] 
\draw   (727,97) -- (743,97) -- (743,113) -- (727,113) -- cycle ;
%Shape: Rectangle [id:dp3444505258816102] 
\draw   (727,113) -- (743,113) -- (743,129) -- (727,129) -- cycle ;
%Shape: Rectangle [id:dp0852649981660425] 
\draw   (727,49) -- (743,49) -- (743,65) -- (727,65) -- cycle ;
%Shape: Rectangle [id:dp9924133663443351] 
\draw   (833,65) -- (849,65) -- (849,81) -- (833,81) -- cycle ;
%Shape: Rectangle [id:dp44917023078204443] 
\draw   (833,81) -- (849,81) -- (849,97) -- (833,97) -- cycle ;
%Shape: Rectangle [id:dp10529894662344819] 
\draw   (833,97) -- (849,97) -- (849,113) -- (833,113) -- cycle ;
%Shape: Rectangle [id:dp2802509291627511] 
\draw   (833,113) -- (849,113) -- (849,129) -- (833,129) -- cycle ;
%Shape: Rectangle [id:dp5216207321547617] 
\draw   (833,49) -- (849,49) -- (849,65) -- (833,65) -- cycle ;
%Shape: Rectangle [id:dp5749837529789577] 
\draw   (937,65) -- (953,65) -- (953,81) -- (937,81) -- cycle ;
%Shape: Rectangle [id:dp04883299519634843] 
\draw   (937,81) -- (953,81) -- (953,97) -- (937,97) -- cycle ;
%Shape: Rectangle [id:dp3411537152276578] 
\draw   (937,97) -- (953,97) -- (953,113) -- (937,113) -- cycle ;
%Shape: Rectangle [id:dp11206511007533004] 
\draw   (937,113) -- (953,113) -- (953,129) -- (937,129) -- cycle ;
%Shape: Rectangle [id:dp6988000837177359] 
\draw   (937,49) -- (953,49) -- (953,65) -- (937,65) -- cycle ;
%Shape: Rectangle [id:dp3539944222247906] 
\draw  [dash pattern={on 4.5pt off 4.5pt}] (39.75,80.6) -- (167.5,80.6) -- (167.5,261.6) -- (39.75,261.6) -- cycle ;

% Text Node
\draw (307,576) node [anchor=north west][inner sep=0.75pt]  [font=\Large,color={rgb, 255:red, 26; green, 66; blue, 114 }  ,opacity=1 ] [align=left] {Avaliação};
% Text Node
\draw (806,569) node [anchor=north west][inner sep=0.75pt]  [font=\Large,color={rgb, 255:red, 26; green, 66; blue, 114 }  ,opacity=1 ] [align=left] {\textcolor[rgb]{0,0,0}{Título}};
% Text Node
\draw (840.2,530.5) node   [align=left] {\begin{minipage}[lt]{47.05600000000003pt}\setlength\topsep{0pt}
	\begin{center}
	[MASK]
	\end{center}
	
	\end{minipage}};
% Text Node
\draw (735.2,530.5) node   [align=left] {\begin{minipage}[lt]{47.05600000000003pt}\setlength\topsep{0pt}
	\begin{center}
	Veio
	\end{center}
	
	\end{minipage}};
% Text Node
\draw (105.2,530.5) node   [align=left] {\begin{minipage}[lt]{47.05600000000001pt}\setlength\topsep{0pt}
	\begin{center}
	[CLS]
	\end{center}
	
	\end{minipage}};
% Text Node
\draw (630.2,530.5) node   [align=left] {\begin{minipage}[lt]{47.05600000000003pt}\setlength\topsep{0pt}
	\begin{center}
	[SEP]
	\end{center}
	
	\end{minipage}};
% Text Node
\draw (525.2,530.5) node   [align=left] {\begin{minipage}[lt]{47.056000000000154pt}\setlength\topsep{0pt}
	\begin{center}
	liga
	\end{center}
	
	\end{minipage}};
% Text Node
\draw (420.2,530.5) node   [align=left] {\begin{minipage}[lt]{47.05600000000003pt}\setlength\topsep{0pt}
	\begin{center}
	não
	\end{center}
	
	\end{minipage}};
% Text Node
\draw (315.2,530.5) node   [align=left] {\begin{minipage}[lt]{47.05600000000003pt}\setlength\topsep{0pt}
	\begin{center}
	tv
	\end{center}
	
	\end{minipage}};
% Text Node
\draw (210.2,530.5) node   [align=left] {\begin{minipage}[lt]{47.05600000000003pt}\setlength\topsep{0pt}
	\begin{center}
	A
	\end{center}
	
	\end{minipage}};
% Text Node
\draw (945.2,530.5) node   [align=left] {\begin{minipage}[lt]{47.05600000000003pt}\setlength\topsep{0pt}
	\begin{center}
	[SEP]
	\end{center}
	
	\end{minipage}};
% Text Node
\draw (497,385) node [anchor=north west][inner sep=0.75pt]  [font=\Large] [align=left] {BERT};
% Text Node
\draw (945.2,226.3) node   [align=left] {\begin{minipage}[lt]{47.05600000000003pt}\setlength\topsep{0pt}
	\begin{center}
	[SEP]
	\end{center}
	
	\end{minipage}};
% Text Node
\draw (840.2,226.3) node   [align=left] {\begin{minipage}[lt]{47.05600000000003pt}\setlength\topsep{0pt}
	\begin{center}
	[MASK]
	\end{center}
	
	\end{minipage}};
% Text Node
\draw (735.2,226.3) node   [align=left] {\begin{minipage}[lt]{47.05600000000003pt}\setlength\topsep{0pt}
	\begin{center}
	Veio
	\end{center}
	
	\end{minipage}};
% Text Node
\draw (105.2,226.3) node   [align=left] {\begin{minipage}[lt]{47.05600000000001pt}\setlength\topsep{0pt}
	\begin{center}
	[CLS]
	\end{center}
	
	\end{minipage}};
% Text Node
\draw (630.2,226.3) node   [align=left] {\begin{minipage}[lt]{47.05600000000003pt}\setlength\topsep{0pt}
	\begin{center}
	[SEP]
	\end{center}
	
	\end{minipage}};
% Text Node
\draw (525.2,226.3) node   [align=left] {\begin{minipage}[lt]{47.05600000000019pt}\setlength\topsep{0pt}
	\begin{center}
	liga
	\end{center}
	
	\end{minipage}};
% Text Node
\draw (420.2,226.3) node   [align=left] {\begin{minipage}[lt]{47.05600000000003pt}\setlength\topsep{0pt}
	\begin{center}
	não
	\end{center}
	
	\end{minipage}};
% Text Node
\draw (315.2,226.3) node   [align=left] {\begin{minipage}[lt]{47.05600000000003pt}\setlength\topsep{0pt}
	\begin{center}
	tv
	\end{center}
	
	\end{minipage}};
% Text Node
\draw (210.2,226.3) node   [align=left] {\begin{minipage}[lt]{47.05600000000003pt}\setlength\topsep{0pt}
	\begin{center}
	A
	\end{center}
	
	\end{minipage}};
% Text Node
\draw (442,13) node [anchor=north west][inner sep=0.75pt]   [align=left] {Masked Language Modelling};
% Text Node
\draw (105,35.5) node  [font=\footnotesize] [align=left] {\begin{minipage}[lt]{52.65401904761903pt}\setlength\topsep{0pt}
	\begin{center}
	\textit{Next }\\\textit{Sentence}\\\textit{Prediction}
	\end{center}
	
	\end{minipage}};


\end{tikzpicture}

	\label{bert_fig}
	\caption{Diagrama ilustrativo do BERT para a task compartilhada de \textit{Masked Language Modelling} e \textit{Next Sentence Prediction}. }
	\centering
\end{figure}

Alternativamente, podemos tentar aproveitar a semelhança do objetivo deste trabalho com a tarefa de \textit{Masked Language Modelling} (MLM) do modelo pré-treinado, que tenta recuperar os tokens mascarados aleatóriamente nas sentenças. Apesar de parecer a melhor alternativa, existem algumas desvantagens aparentes: no objetivo de geração de texto deste trabalho, vamos sempre adicionar o token \texttt{[MASK]} no final da segunda sentença, que estará sempre seguido do token final \texttt{[SEP]}. Naturalmente, isto irá fazer o modelo original acreditar que o token mascarado é sempre um token próximo ao final de sentença (um caractere de ponto final, por exemplo) e irá prejudicar a performance do texto que iremos gerar de maneira causal autoregressiva. Ao fazermos o \textit{fine tuning} segundo esta abordagem, estamos grosseiramente tentando converter um modelo treinado com a tarefa de \textit{Masked Language Modelling} para um modelo causal de sumarização de textos.

É difícil de prever de antemão se o BERT-MASK produzirá resultados melhores ou piores que o BERT-CLS (proposto originalmente pelo enunciado da tarefa), então, a título de curiosidade, decidi avaliar a performance destas duas abordagens diferentes.

\section{Treinamento}

\subsection{Encoder-Decoder BiLSTM}

O treinamento da encoder-decoder BiLSTM é feito utilizando a técnica \textit{teacher forcing} (\textcite{williams1989learning}), que passa a sequência correta inteira de tokens para o decoder que tenta prever o próximo token relativamente a cada item da sequência passada. Essa técnica na prática acelera a convergência de modelos sequenciais e usualmente também facilita a implementação destes modelos.

Treinamos o modelo com \textit{early-stopping} em uma amostra de 72\% do conjunto total de dados, validando o modelo no final de cada época em um conjunto de validação que corresponde a 8\% dos dados. O treinamento do modelo é interrompido se o custo calculado no conjunto de validação não diminuir em 5 épocas consecutivas. Quando o treinamento é interrompido, apenas o modelo com menor custo no conjunto de validação é salvo.

\subsection{BERT-CLS e BERT-MASK}

Seguindo o procedimento de preparação dos dados descrito no enunciado do exercício, utilizamos a mesma amostra de 80\% dos dados extraidos para o treinamento da LSTM. Ao dividir a geração de texto token a token, o processamento destes dados produz uma amostra com mais instâncias. Por limitações computacionais, utilizei apenas 2/3 destes dados durante apenas uma única época de treinamento. Por sua vez, como a função de custo foi calculada apenas em dados que não foram antes vistos pelo modelo, não foi necessário utilizar \textit{early-stopping} para previnir \textit{overfitting}.

O modelo BERT-CLS convergiu de maneira estável, sem precisar de um tamanho de \textit{batch} muito elevado, finalizando após cerca de 20h de treinamento no \textit{Google Colab}. O modelo BERT-MASK que exigiu maior esforço de treinamento e teve convergência difícil, o que é justificável, visto que alteramos a tarefa base em que o modelo foi treinado originalmente. Para garantir estabilidade no treinamento do BERT-MASK foi necessário utilizar um tamanho de \textit{batch} elevado por meio da técnica de \textit{gradient accumulation}, para não exceder a capacidade da infraestrutura disponível. A configuração exata do treinamento segue reproduzida na tabela a seguir



\newpage

\section{Resultados experimentais}

Neste trabalho, foram feitos 6 experimentos diferentes segundo as combinações dos hiperparâmetro da camada de \textit{Dropout} (0\%, 25\% e 50\%) entre as duas modalidades de LSTM (unidirecional e bidirecional). Os resultados registrados em treinamento nos conjuntos de validação e treino são apresentados nas Figuras 2 e 3. 

Verificamos dos gráficos de treinamento que os modelos de LSTM unidirecionais apresentaram problemas graves de \textit{underfitting}, que impediram o custo de diminuir significativamente em 100 época de treinamento. Isso é esperado, dado que foram feitas simplifcações no modelo, como utilizar uma dimensão baixa na camada de \textit{embeddings} e congelar o treinamento dos mesmos. Uma alternativa para aumentar a complexidades destes modelos seria incluir mais uma camada recorrente unidirecional logo após a primeira.  

Para as LSTMs bidirecionais, observamos o comportamento de overfitting descrito no enunciado. Após cerca de 25 épocas, o custo de validação começou a crescer para os três modelos. Verificamos também que a presença de \textit{Dropout} reduziu a diferença entre o custo de treinamento e o custo de validação, conforme o esperado.


\printbibliography

\end{document}